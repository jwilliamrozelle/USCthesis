\section{Introduction}

Primary health coverage is essential to building and maintaining progress on morbidity and mortality reductions, particularly in Low-and Middle-Income Countries (LMICs) where preventable deaths still rank among the most common <CITE>. A robust body of evidence has identified health service quality as a core ingredient for achieving better health outcomes \autocite{world_health_organization_delivering_2018}, but treatment success is conditional on service utilization - a construct that is largely determined by accessibility and perceptions of quality <CITE>.

\subsection{WHY IT MATTERS} %

In higher income countries, previous research has shown a relationship between client satisfaction and health service utilization \autocite{zastowny_patient_1989} <CITE MORE>. More satisfied individuals are more likely to seek care from providers. A more thorough avenue of determinants of health service utilization may offer a path to higher health service utilization.

\subsubsection{PATIENT SATISFACTION GENERALLY}

\subsubsection{PATIENT SATISFACTION IN LMICs}




\section{Methods}

\subsection{Data}

Data will be SPA data from DHS Phase VII with client exit interviews. Afghanistan, Haiti, Nepal and Tanzania meet this eligibility criteria and were the surveys were conducted between 2015 and 2018.

\subsection{Outcomes}

There are several measures of client (alternatively referred to as "patient") satisfaction available in the Service provision assessment. Each of these indicators are separately available for the three health services (antentatal care, sick child care, and family planning). For this analysis, I extract two composite indicators as key outcomes. I construct composite indicators, because in these settings, few clients report complaints or express dissatisfaction. Composite indicators enable exploration facilitate a measure of \textit{any} dissatisfaction.

\subsubsection{Client satisfaction} Two questions are used to construct this indicator. First, for each service type, the questionnaire asks "how satisfied are you with the services you recieved today." Possible responses include "very satisfied", "more or less satisfied" or "not satisfied". Second, respondents are asked whether they would recommend the facility to a friend or family member (possible responses are "yes" or "no"). If a client does not respond that they are "very satisfied" or that they would not recommend the facility to a friend, client satisfaction is coded 0, otherwise it is coded 1.

\subsubsection{Free of problems}: Clients are asked about 11 specific potential problems during their visit and whether each was a "major", "minor", or "no problem" during their visit. Items such as wait time, cost, privacy, respect, etc. The few peer reviewed studies that have examined this based on SPA data have collapsed this into a dichotomous free of problems (=1) vs. any problem (=0) problems \autocite{do_quality_2017, bergh_identifying_2022}. One relevant dissertation coded this as an additive index with values possible values between 0 and 11 \autocite{riese_associations_2019}.
    
\end{itemize}

\subsection{Independent Variables}

Several variables measuring facility level, provider level, and client demographic level, and characteristics of the provider's interaction with the client are explored as potential predictors of facility satisfaction.

\subsubsection{Service Readiness Index}

Using a dimensional reduction technique such as Principal Components Analysis (PCA), Multiple Correspondence Analysis, or Latent Class Analysis could be helpful for creating a more useful index following an approach similar to DHS wealth index calculation \autocite{dhs_program_dhs_nodate}. However, the fact that a substantial proportion of clients report satisfaction on all measures may constrain the possible analytic approaches.

\subsubsection{Provider characteristics}

\subsubsection{Client demographic characteristics}

\subsubsection{Visit observation}

\subsection{Analytic Approach}

I will likely explore a variable selection technique (such as stepwise regression) to explore determinants that are meaningfully associated with measures of satisfaction in selected countries. The initial list of included predictors will be from the service readiness index \autocite{world_health_organization_service_2014}. Regression analysis will be performed using a Bayesian framework, and I will explore the utility of each outcome.